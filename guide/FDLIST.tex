\documentclass[12pt]{scrartcl}

\usepackage[british]{babel}
\usepackage[dvipsnames,svgnames]{xcolor}
\definecolor{darkblue}{rgb}{0.0,0.0,0.6}
\usepackage[pdflang=en-UK, colorlinks, urlcolor=darkblue, linkcolor=darkblue, citecolor=darkblue]{hyperref}

\usepackage[utf8]{inputenc}
\usepackage[T1]{fontenc}
\usepackage{ae}
\usepackage{aecompl}
\usepackage{lmodern}
\usepackage{csquotes}
\usepackage[babel=true]{microtype}
\usepackage{ellipsis}
\usepackage{combelow}
\usepackage{enumitem}

\usepackage{textcomp}
\usepackage{xparse}
\ExplSyntaxOn
\RenewDocumentCommand{\texttt}{m}
 {
  \tl_set:Nn \l_nemgathos_upquotes_tl { #1 }
  \tl_replace_all:Nnn \l_nemgathos_upquotes_tl { '' } { \textquotedbl }
  \tl_replace_all:Nnn \l_nemgathos_upquotes_tl { `` } { \textquotedbl }
  \tl_replace_all:Nnn \l_nemgathos_upquotes_tl { ' } { \textquotesingle }
  \tl_replace_all:Nnn \l_nemgathos_upquotes_tl { ` } { \textquotesingle }
  { \ttfamily \tl_use:N \l_nemgathos_upquotes_tl }
 }
\tl_new:N \l_nemgathos_upquotes_tl
\ExplSyntaxOff

\usepackage{amsthm}
\theoremstyle{definition}
\newtheorem*{warn}{Warning}

\title{FD-Conferences: Maintainer's Guide}
\date{\today}

% ==== ==== ==== ==== ==== ====
\begin{document}
% ==== ==== ==== ==== ==== ====

\maketitle

\tableofcontents

\section{Introduction}

This document is intended as a rough guide for maintaining the list of conferences on the \href{https://fdlist.math.uni-bielefeld.de/}{FDLIST}, with instructions for common tasks, an outline of the FDLIST's house style, and some comments on potentially surprising or confusing features.
It does not contain full documentation of the implementation of the conference list.

\section{Initial setup and basic commands}

While the main FDLIST site is a Discourse page located at
\begin{center}
\verb|https://fdlist.math.uni-bielefeld.de/|
\end{center}
the list of conferences is hosted separately at
\begin{center}
\verb|https://gjassoah.github.io/fd-conferences/|
\end{center}
and embedded into the Discourse site as an iframe.
This setup allows most updates (e.g.\ removing a conference from the list of upcoming events once it has happened) to be automated, with the actual HTML files for the page generated by the software Hugo and re-compiled automatically once per day (or whenever an edit is made).
As a result, the conference list is not edited using Discourse's GUI, but by editing this separate GitHub page, or more precisely, the database used by Hugo to generate it.

\subsection{Cloning}
To get started, you will need to have a \href{https://github.com}{GitHub} account, and have Git installed on your machine: the instructions for achieving this second part will depend on your operating system.
In this guide we will assume that you interact with Git through the command line, and not through a GUI. You will also need to contact \href{mailto:gjasso@math.uni-koeln.de}{Gustavo Jasso} with your GitHub username, so that you can be given access to the fd-conferences repository.

Now you can create a local copy of the database by navigating to the folder where you want to keep this and running
\begin{center}
\verb|git clone https://github.com/gjassoah/fd-conferences.git|
\end{center}
By default, this will download the database to a folder named `fd-conferences' under the current directory, but this folder name can be safely changed after the fact without causing trouble.

\subsection{Basic Git usage}
\label{s:basic_Git}
The specifics of creating and editing events are handled below, but first we briefly recall the basic Git commands you will need.
If you create a new file (most commonly an event listing) that you want Git to track, run
\begin{center}
\noindent\verb|git add |(path to file)
\end{center}
from the fd-conferences directory.
In practice, files you add will essentially always be under the `content' subdirectory, in which all files are tracked, so it is safe to run
\begin{center}
\verb|git add content/|
\end{center}
for efficiency.

To save changes, run
\begin{center}
\verb|git commit -am "|(message)\verb|"|
\end{center}
where (message) describes briefly your changes.
(The \verb|-a| flag prevents you from having to manually \verb|git add| files which have only been \emph{modified} rather than created, and the \verb|-m| flag lets you add the commit message immediately rather than being prompted for it afterwards.)

To synchronise your local copy of the database with that on GitHub, run
\begin{center}
\verb|git pull|
\end{center}
to import changes from the GitHub copy to your local copy, and then
\begin{center}
\verb|git push|
\end{center}
to export your local changes to the remote copy.
\textbf{Always pull before you push!}

Very occasionally you may encounter merge conflicts which need to be resolved manually.
For this you will need to use Git's messages to work out what needs to be done.

\subsection{Contact details}
Another important detail when you become the new maintainer is to update the contact details on the FDLIST homepage.
To do this you need to sign up to the FDLIST (if, shockingly, you have not already done so) and get one of the existing moderators to make you an admin.
You will then have the ability to edit Discourse pages: click the pencil icon in the upper right of the homepage to change the contact email for conference announcements to your own.

\section{Events}
Information concerning each event on the FDLIST is stored in a markdown file in the directory
\begin{center}
\verb|content/events/|
\end{center}
To create a new event, we recommend beginning by making a copy of the file \verb|template.md|, a mostly blank version of the event file with the most common fields listed, and renaming it to something suitable.
The standard file naming convention (since the relaunch of the site in 2021) is
\begin{center}
\verb|YYYY-MM-DD-|(short name)\verb|_|(city)\verb|.md|
\end{center}
which is convenient for sorting and for finding events later, but anything that works for you (and ideally future users!)\ is fine.

\begin{warn}
While it can be sensible to update \verb|template.md| periodically (especially the \verb|dateadded| field), this file is tracked by Git, and so the \verb|enddate| field must remain set to a date before 2000-01-01 to avoid creating a blank line at the place on the site where this `event' would be printed.
An alternative would be to leave the file untracked (and add it to \verb|.gitignore| so that it is not re-tracked when running \verb|git add content/|), but then it would vanish from the remote database, which is inconvenient when cloning the repository to a new location.
You could instead decide to move it to a different folder.
\end{warn}

Before describing the various fields that can be set in the markdown file, we point out that the Hugo implementation has been set up so that, for the most part, fields can be safely left empty without unintended consequences.
For example, if an event has no speakers (yet), one can simply leave the line \verb|speakers = ""| in the markdown file, and this section of the listing will simply not appear when the site is compiled (in contrast to seeing `Speakers:' followed by an empty space).

A field which is not specified in the markdown file at all is also treated as empty. This makes it straightforward to create new fields if needed: they do not need to be added (with an empty value) to all previous events.

\begin{warn}
Line breaks (a.k.a.\ carriage returns) are not permitted in any field, and will cause fatal errors when Hugo compiles the site.
Watch out for these, since it can be easy to import them accidentally when copying text from an email.
\end{warn}

\subsection{Standard fields}
\label{s:standard_fields}
The following standard fields are present in the template file, and are expected to be needed for most events.
Most are self-explanatory.
The most obvious questions of style are addressed here, but further comments are found in Section~\ref{s:style} below.

\begin{warn}
In most of the fields below, the typewriter apostrophe \texttt{'} will be replaced by a `proper' apostrophe / right quotation mark \texttt{’} when the page is rendered. If you ever want a \emph{left} quotation mark \texttt{‘}, this must be entered manually to avoid this auto-correction.
There is no auto-correction for double quotation marks, but you are still encouraged to enter these in their left and right variants, rather than as \texttt{"}.
A similar rule replaces `~-~' (space ldash space) by the more typographically correct `~--~' (space ndash space), except in dates, where the dash is expected to indicate a range rather than a break, and so the extra spaces are also removed.
\end{warn}

\noindent\verb|title|\nopagebreak

The name of the event.
\medskip

\noindent\verb|date|\nopagebreak

The date that the event starts, in YYYY-MM-DD format.
\medskip

\noindent\verb|enddate|\nopagebreak

The date that the event ends, in YYYY-MM-DD format.
\medskip

\noindent\verb|dates|\nopagebreak

The dates of the event \emph{in text format}.
The FDLIST style is `Month dd--dd, yyyy', where the month is written as a full word, and the days are written with one or two digits as appropriate (i.e.\ without leading zeros).
If an event falls across two months, the format `Month dd--Month dd, yyyy' is used.
Very occasionally an event falls across two years, in which case use `Month dd, yyyy--Month dd, yyyy'.
While these formats use the character `--' (ndash, written \verb|--| in \TeX), the string `~-~' (space ldash space) can be used instead in the markdown file, and will be automatically replaced when the page is rendered by Hugo.

\begin{warn}
The fields \verb|date| and \verb|enddate| are used by Hugo to decide where on the site the event should appear, while \verb|dates| is the text string which is actually printed on the page.
There is nothing in the implementation to alert the maintainer if these dates are not consistent with each other!
\end{warn}

\noindent\verb|dateadded|\nopagebreak

The date that you added the event to the FDLIST (so that the `NEW' flag will be displayed on the site if 28 days or fewer have passed since this date).
\medskip

\noindent\verb|location|\nopagebreak

The location of the event, usually in the format `Venue, (City,) Country', where the city is omitted if it is clear from the name of the venue.
(So, for example, `University of Bielefeld, Germany', but `NTNU, Trondheim, Norway'.)
If there are multiple locations, separate them by forward slashes.
\medskip

\noindent\verb|webpage|\nopagebreak

The URL of the event's webpage.
\medskip

\noindent\verb|organisers|\nopagebreak

The list of organisers of the event.
\medskip

\noindent\verb|speakers|\nopagebreak

The list of speakers of the event, in (English) alphabetical order by family name as much as possible. Generally we only correct blatant errors in ordering: for example, if the organisers' ordering makes sense up to deciding which name is a person's family name, leave it as it is.
Speakers to be confirmed should be indicated with an asterisk (without any further explanation). If the speakers fall into two different categories, you can use additional fields to indicate this, as explained in the next subsection.

\subsection{Extra fields}

There are some additional fields which are not present in the template file, but which can be added if necessary.
\medskip

\noindent\verb|note|\nopagebreak

A very flexible field which adds some additional information directly under the title of the event.
Most commonly used for `in honour of...', or `a satellite meeting of...'.
\medskip

\noindent\verb|warning|\nopagebreak

Adds a warning before the title (in the same style as `NEW') using the text string provided, for example `POSTPONED' or `CANCELLED'.
(Hopefully this won't be needed much, but see 2020 in the archive for many examples.)
The all-caps style is not automatic, and so the text should be entered into the markdown file in all-caps.
\medskip

\noindent\verb|committee|\nopagebreak

The scientific committee of the event.
\medskip

\noindent\verb|special_talks|\nopagebreak

When an event has a special category of talks (e.g.\ mini-courses, lecture series), the speakers in this category can be listed separately.
This field specifies the name of this special category of talks.
\medskip

\noindent\verb|special_speakers|\nopagebreak

The speakers in the special category of talks (who should not be written additionally in the \verb|speakers| field, unless they are giving both an ordinary talk and a special talk).

\begin{warn}
If \verb|special_speakers| is set but \verb|special_talks| is blank, the special category of talks will be given the default name `Lecture Series'.
\end{warn}

\noindent\verb|schedule|\nopagebreak

Rarely used: a URL pointing to the schedule of the conference.
The main use case for this field was to give a link to schedules for online meetings on \verb|researchseminars.org| (with its automatic time zone adjustment).
It is not recommended to add schedule links for in-person events where the schedule is hosted on the usual webpage.

\subsection{The archive}
As the pages are recompiled daily, events will automatically move to the archive as and when they happen, without any user involvement necessary.
If corrections are needed to an event in the archive, this is also done in exactly the same way as one would modify an upcoming event, by modifying its markdown file in the database.
A new page will be added to the archive on January 1st every year up to and including 2099---in the unlikely event that you are reading this in 2099, you will need to extend this deadline.

To do this, go to \verb|content/archive| and add some more text files named \verb|yyyy.md|, with the content
\begin{center}
\verb|{"date": "yyyy-01-01 00:00:00"}|
\end{center}
for as many values of \verb|yyyy| as you feel is necessary.

Theoretically, it is possible to add historical information pre-dating the founding of the FDLIST in 2000, which would require adding earlier years in a similar way.
Beware that this might also require modifying the dates in the event template to prevent it from contributing a blank line to the archive.

\section{Style guidelines}
\label{s:style}
The FDLIST has a traditionally had a laissez-faire approach to style, to avoid lengthy editing when copying information you have been sent by email.
However, we have tried to establish a few general style principles to create a little extra consistency, when this does not lead to significant workload.
As might be expected from a long running database which has had several different maintainers, events in the archive may not be entirely consistent with these style guidelines, but can be updated to follow them if you notice a discrepancy and wish to change it.

The guidelines on how to write names of people and places, covered in Sections~\ref{s:people} and \ref{s:places}, should quickly become intuitive despite the length, since the special cases explained all follow from a small number of general principles.
Date formats were covered already in Section~\ref{s:standard_fields}, so we do not repeat these here.

\subsection{What to include}
If you are asked to add an event to the FDLIST, this request should only be turned down if the event is \emph{very} clearly off topic, which happens at the extremely low frequency of approximately once per decade.
Historically we have included events with a diverse set of primary subjects, as long as one of the organisers of the event believes there is value in advertising to the FDLIST community.

You should feel free to add events you feel are relevant without being asked to by an organiser, if the details of the event are publicly available. Similarly, you can update event listings based on their webpages without being prompted.

Invitation-only events (such as Oberwolfach meetings) are often overlooked since the organisers do not feel the need to advertise, but it is preferable if they are also added to the FDLIST in order to maintain an accurate record of historical activity.
You may need to prod an organiser of such an event to provide details such as a list of speakers if this information is hidden from you.

\subsection{Non-rules}
\label{s:non-rules}
In keeping with the generally relaxed style principle, here are several aspects on which no consistent style is maintained, and we typically follow whatever the event organisers do.

\begin{itemize}
\item Titles of events can be entered in either sentence case or title case.
\item Lists of organisers, speakers and committee members can be entered with or without affiliations (although there is a general preference for using commas to separate different speakers, and entering affiliations in brackets).
\item Names of venues can be entered in either English or the local language (e.g.\ `Universität Bielefeld' or `University of Bielefeld'), although they should be written in a Latin-based script. Note that several U.K.\ universities feel strongly about whether their names are written `University of $X$' or `$X$ University', so you should avoid switching between the two formats, although you can typically trust what you are told by event organisers.
\item The same applies to names of cities (e.g.\ all of Cologne / Köln, Munich / München, Padua / Padova are acceptable).
Be careful, however, to avoid dropping accents and diacritics unless this really corresponds to a commonly used English name for the city.
For example, Istanbul and Montreal are standard English versions of İstanbul and Montréal respectively (with their own pronunciations), but Dusseldorf is just a misspelling of Düsseldorf (and Munster is a province of Ireland, whereas Münster is a city in Germany).
If in doubt, keep the accents!
\item Cities in the U.S.A.\ or Canada are sometimes also given their state or province (using the standard two-letter abbreviation), but this is not necessary.
\item The specificity of the venue (e.g.\ whether it includes just the name of a university or of an actual building / department) is flexible.
\end{itemize}

\subsection{Names of people}
\label{s:people}
When writing names of people (and places, see Section~\ref{s:places}), there is a general preference for typographical correctness, unless the person themselves \emph{very} consistently writes their name in an `anglicised' way.

\subsubsection{Latin-based scripts}
For names in Latin scripts, you should usually attempt to preserve all accents and/or diacritics (to avoid replacing a character by a different valid character from the same language), and even to re-instate these if they have been omitted by the event organisers. For example: İlke Çanakçı, Jan Šťovíček, Bertrand Toën (not Ilke Canakci, Jan Stovicek, Bertrand Toen ... and especially not Bertrand Töen).

It is common for Spanish-speaking mathematicians, especially from Latin America, to consistently drop diacritics in their name on papers, and this should be respected where it happens (along with the choice of which subset of names to write).
For example: Ana Garcia Elsener, Hipolito Treffinger (not Ana García Elsener, Hipólito Treffinger).
However, as an exception to the exception (!), organisers will sometimes re-instate these accents when an event is primarily in Spanish, in which case we usually follow this on the FDLIST as well.

\begin{warn}
It is good to be aware of some potentially confusing characters.

In Czech, the characters ď and ť are not written with apostrophes: the apostrophe-like element is a truncated háček, with different spacing: in capitals, these characters become Ď and Ť.
Compare Šťovíček to Št'ovíček, for example.

The German Eszett (ß) is equivalent (by law!)\ to `ss', and both can be used (e.g.\ Christof Geiß or Christof Geiss).

Faroese and Icelandic include the runic characters ð and þ (or Ð and Þ in upper case), but the alphabets are still primarily Latin-based, and so these should not be transliterated.

In the Hungarian characters ő and ű (e.g.\ in Balázs Szendrői), the accent is a double-acute, not an umlaut (as in ö and ü, which are also used in Hungarian to represent different sounds).

Romanian includes the characters \cb{s} and \cb{t}, which often get rendered incorrectly (even in Romania) as ş and ţ:  the diacritic is a comma below, not a cedilla.
On the other hand, the character ş \emph{is} correct in Turkish.
\end{warn}

\subsubsection{Non-Latin scripts}
Names in non-Latin scripts should be transliterated to Latin characters for wider readability.
Generally one should follow the transliteration used by the person in question, which may differ from the official one (e.g.\ Mikhail Gorsky, not Mikhail Gorskii).

Transliterations of Chinese names usually do not include the tone marks present in formal pīnyīn (so Yu Qiu, not Yŭ Qiū), and transliterations from Japanese typically drop the macrons indicating long vowels (Ryo Kanda, not Ryō Kanda).
On the other hand, if this extra information is provided by the event organisers, there is no need to remove it, and it should be present if typically used by the person in question.

\subsubsection{Transgender people}
When a transgender person changes their name to better reflect their gender identity, the default assumption is that this name change should be reflected in \emph{the entire FDLIST archive}, not only in future events.
You may wish to check with the person concerned that this is their actual preference, if you feel able to do so sensitively.
%It is increasingly easy to get journals to update names on previously published articles in this situation, which can also reveal a preference.

When people change names for other reasons, it is not usually necessary to change all existing occurrences in the database.
Use your best judgement, and follow the preference of the person concerned if you know it.

\subsubsection{Special cases and common errors}
\begin{itemize}
\item Lidia Angeleri Hügel is sometimes written with a hyphen (`Angeleri-Hügel'), but should not be, to the best of my knowledge.
\item Steffen Koenig has written his name consistently for many years with the transliteration `oe' in place of `ö', and this should be respected (even when event organisers try to `fix' it).
%\item A short list of Czech representation theorists, with apologies for any omissions: Pavel Příhoda, Jiří Rosický, Pavel Růžička, Jan Šťovíček, Jan Šaroch, Jan Trlifaj, Jan Žemlička (spot the odd one out!).
%\item In Dutch, some importance is placed on the capitalisation of the phrase `van', and similar, in names.
%The most common is all lower case, but there are notable exceptions, such as Michel Van den Bergh.
%It is best not to overrule what you have been given by the event organisers unless you are sure about what you are doing.
\end{itemize}

\subsection{Names of places}
\label{s:places}
For writing names of places, follow the general guidance from Section~\ref{s:people} for writing names of people.
As already mentioned in Section~\ref{s:non-rules}, names of venues and cities can be given either in English or in the local language (transliterated to Latin characters).
This guideline also applies to affiliations in lists of organisers and speakers.

\subsubsection{Names of countries}
Countries should typically be named in English.
\begin{itemize}
\item Where the name of a country is often referred to by an abbreviation, this should be used, and written with periods.
That is, `U.A.E.', `U.K.' and `U.S.A.', not `UAE', `UK' and `USA' or `United Arab Emirates', `United Kingdom' and `United States (of America)'.
\item England, Scotland, Wales and Northern Ireland are (at the time of writing!)\ constituent parts of the U.K., and so should not be written instead of, or in addition to, `U.K.'.
\item Some countries have recently expressed a preference for how their names should be rendered in English (e.g.\ at the U.N.)\ that differs from how they were written historically: we should try to respect this. Notable examples are Czechia (not Czech Republic, since 2016), Eswatini (not Swaziland, since 2018) and Türkiye (not Turkey, since 2022).
\end{itemize}

\subsubsection{Online events}
For fully online events, the location can be given simply as `Online', but usually there is still an obvious hosting institution: in this case, use `Online / (host)'. For example, use `Online / NTNU, Trondheim, Norway' for the annual NTNU flash talks.
If in doubt, check with the event organisers.

For hybrid events, we typically list only the physical location.

\subsubsection{Common locations and special cases}
Here is a very incomplete list of locations including diacritics that are often dropped (or not including some which are often added!).
\begin{itemize}[noitemsep]
\item University of Almería, Spain
\item Universitat Autònoma de Barcelona, Spain
\item Babe\cb{s}-Bolyai University, Cluj-Napoca, Romania
\item Doğuş University, Istanbul, Turkey
\item Jagiellonian University, Kraków, Poland
\item UQAM, Montreal, Canada\footnote{While the full name of the university is Université du Quebec à Montréal, the university itself drops the grave accent on `à' in its all-caps abbreviation. The English `Montreal' could also be written as the French `Montréal'.}
\item Université de Neuchâtel, Switzerland
\item Università La Sapienza, Rome, Italy\footnote{Università is sometimes rendered as Universita' (with an apostrophe) in email for convenience, but we should use the proper character.}
\item Nicolaus Copernicus University, Toruń, Poland
\item ETH, Zürich, Switzerland
\end{itemize}

Several universities in London are named `$X$ College London', with no comma after the word `College': this can look like a typo if you are unfamiliar with it.

\section{Hugo implementation}

You may at some stage want to modify the way in which Hugo renders the conference list.
For this you will be mostly left to explore and experiment on your own, but here at least are some small pointers.
The key files have a \verb|.html| extension, but mix HTML and Hugo syntax.
When compiled, the Hugo commands are expanded into HTML, constructed recursively depending on the markdown files describing each event (or year in the archive).

The most relevant files are under \verb|layouts/|, and especially \verb|layouts/partials/|.
The layout of the list of upcoming meetings is controlled by
\begin{center}
\verb|layouts/partials/conference-list.html|
\end{center}
with the layout of individual events controlled by the section beginning with
\begin{center}
\verb|<div class="event">|
\end{center}
This is the part that you should modify if you want to add new functionality, for example new fields to be used in constructing each event listing.
We will not try to give a tutorial in Hugo syntax in this document, but point out that the copious use of
\begin{center}
\verb|{{ if .Params.|(field)\verb| }}|
\end{center}
is what checks whether a field has actually been set before attempting to do anything with the contents, so that sections in the event listing are simply omitted if there is no relevant data.

Note that the layout of events in the \emph{archive} is controlled instead by the file
\begin{center}
\verb|layouts/partials/event/listing.html|
\end{center}
This is a bit of a historical artifact, but it does allow us to let these two layouts deviate from one another: for example, a meeting on the archive page will never be marked as `NEW'. Remember, however, that if you want to add new features or fix a bug both for upcoming events and in the archive, then both files need to be modified.

The code for the upcoming meetings page is at
\begin{center}
\verb|layouts/index.html|
\end{center}
but is very minimal, since it essentially just imports other files.
The code used to generate pages for the archive is at 
\begin{center}
\verb|layouts/archive/single.html|
\end{center}
The page style throughout the site is set by CSS, with the style file at
\begin{center}
\verb|static/css/style.css|
\end{center}
This should only be modified with care, since the style has been set to closely match that of the surrounding Discourse page into which the conference list is embedded.

\section{Potential improvements}
The purpose of this final section is to indicate some potential features or improvements that have not been implemented due to lack of time, knowledge or energy.
They are listed in no particular order.

\begin{enumerate}
\item Have Hugo read the date currently entered manually under \verb|dateadded| from the creation date of the file.
(This is the date used to decide when an event is `NEW', and so it is the creation date and not the last modified date that is relevant.)
It may still be useful to allow the maintainer to overrule this if necessary.
\item Have Hugo convert \verb|date| and \verb|enddate| into a text string date, in the correct format, so that \verb|dates| does not need to be set manually.
Again, the option to override would be good.
This functionality was actually lost when moving from the old database hosted at NTNU to the new implementation through Discourse and GitHub.
\item Add a \verb|special_organisers| field analogous to \verb|special_speakers| (e.g.\ for `local organisers').
This is relatively easy, but the number of use cases quite low, and we should beware of letting event listings become extremely long.
Similarly, one could change the implementation to allow for more than one special category of speakers or organisers (but this is not likely to be used often, if at all).
\item Automatically strip out all line breaks in fields in the markdown files, either in the files themselves or when loading the data into Hugo. Again, this functionality existed in the old NTNU database but was lost in the new implementation.
\item Set up a much more substantial list of automatic replacements, with some better implementation than nesting \verb|{{ replace }}| commands, to automate some of the style consistency guidelines (e.g.\ by replacing `UK' by `U.K.'). Beware that the most obvious implementations would make it very difficult for the maintainer to overrule the replacement if needed.
\end{enumerate}

\end{document}
